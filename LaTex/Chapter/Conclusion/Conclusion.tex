\section{Fazit} \label{sec:Conclusion}
Das Modell der RDF kann erfolgreich für die Kalibrierung der radialen Verzerrung von einfachen Linsensystemen genutzt werden. Es gilt jedoch die Voraussetzung einer korrekten Perspektive, welche den Weltpunkt, Bildhauptpunkt und das Projektionszentrum auf dem selben optischen Strahl abbildet. Für Objektive die eine sehr geringe Brennweite und folglich eine extreme radiale Verzerrung abbilden sind laut Literatur~\cite{WangRaddist} andere komplexere Modelle interessanter. Die manuelle Positionierung der Koordinatenpaare kann unter Umständen größere Abweichungen vom erwarteten Abstand hervorrufen, welche zu mangelhaften Korrekturen der radialen Verzerrung in einigen Bildbereichen führen kann. 

Zusammenfasst ist dieses Modell nur unter den idealen Bedingungen zuverlässig und es muss in der Praxis mit Ungenauigkeiten gerechnet werden. Aus diesem Grund ist es für eine robuste Korrektur sinnvoll auch andere Modelle zur Korrektur der radialen Verzerrung zu prüfen.
\section{Aufteilung der Dokumentation}
\begin{table}[H]
		\begin{tabular}{|p{0.45\textwidth} | p{0.55\textwidth}|} 
			\hline
			\textbf{Team Mitglied} & \textbf{Verfasste Abschnitte}\\ \hline
			Artjom Schwabski& Kapitel \ref{sec:Abstrakt}, \ref{sec:Materials}, \ref{sec:Auswertung}; Abschnitt \ref{sec:Wertpaare}, \ref{sec:PluginKlassen} \\ \hline
			Vera Brockmeyer & Kapitel \ref{sec:Introduction}, \ref{Methode}, \ref{sec:Conclusion}, \ref{sec:Auswertung}; Abschnitt \ref{sec:Problem}, \ref{sec:Minimierung} \\ \hline
		\end{tabular}
		\caption{Aufteilung der Dokumentation}
\end{table}

























