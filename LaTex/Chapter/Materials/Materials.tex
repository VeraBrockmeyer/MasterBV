%!TEX root = ..\..\main.tex

\section{Materialien}\label{sec:Materials}
\todo[inline, color=red]{XXX}

\subsection{Hardware}\label{sec:Hardware}
\subsection{Software}

\subsubsection{Klassen}
\todo[inline, color=red]{Artjom}

Im folgenden werden die Methoden der einzelnen Klassen erläutert. Die vollständige UML zur besseren Verständlichkeit der Klassenbeziehungen ist der Abb. \ref{img:UML} zu entnehmen.

\paragraph{point\_grid\_radial\_affin\_distor\_}
Hauptklasse der Anwendung. Implementiert das Interface \emph{PluginFilter} um über ImageJ aufgerufen werden zu können.

Die Klasse besitzt folgende Methoden und deren Funktion:

\begin{table}[H]
\begin{tabular}{p{0.3\textwidth} | p{0.7\textwidth}} 
run & Main-Methode des PlugIns in der die Optimierung aufgerufen wird\\ \hline
setup & Konstruktor-Methode des PlugIns in dem die Bildreferenz gespeichert wird\\ \hline
readData & Liest aus einer in ImageJ geöffneten Textdatei Punkt-Paare ein für Start- und Ziel-Koordinten\\
computeDrawRadialTransformation & \\ \hline
drawTargets & Zeichnet Punkte an den übergebenen Ziel-Koordinaten in das übergebene Bild\\ \hline
computeDrawAffineTransformation & \\ \hline
computeRadius2Center & Berechnet anhand der Parameter den Abstand zum Gittermittelpunkt\\ \hline
compute\_radial\_dist\_koeff & Berechnet mit dem LevenbergMarquadt Optimierer die Koeffizienten der Radialen Verzerrung der übergebenen Punkt und gibt die Koeffizienten zurück\\ 
\end{tabular}
\caption{Methoden der point\_grid\_radial\_affin\_distor\_ Klasse}
\end{table}
\paragraph{SimplePair}
Eine Einfache Klasse zum Speichern der Vorgabe- und Ziel Koordinaten und des Abstandes zum Mittelpunkt.

\paragraph{RadialDistFunction}
Klasse zum Erzeugen der Funktionen für den Optimierer.

\begin{table}[H]
\begin{tabular}{p{0.3\textwidth} | p{0.7\textwidth}} 
RadialDistFunction & Konstruktor der Klasse. Es wird ein SimplePoint Array erwartet welcher Koordinaten-Paare für Start- und Ziel-Koordinaten enthält.\\ \hline
realTargetPoints & Gibt ein Array aus welches nur die Ziel-Koordinaten enthält. Dieses wird für den Optimierer benötigt.\\ \hline
retMVF & Funktion zur Modellierung der Radialen Verzerrung für den Optimierer. BErechnet zu den Vorgegeben Koeffizienten und einer Start-Koordinate die Ziel-Koordinate\\ \hline
retMMF & Jacobi-Matrix-Funktion zur Berechnung der Ableitung nach den einzelnen vom Optimierer vorgegebenen Koeffizienten \\ 
\end{tabular}
\caption{Methoden der RadialDistFunction Klasse}
\end{table}

\begin{figure}[H]
\center
\includegraphics[width=\textwidth]{Images/UML.JPG}
\caption{UML Klassendiagramm}
\label{img:UML}
\end{figure}


\paragraph{UnwrapJ}
 ist ein für ImageJ entwickeltes Plugin, das die elastische Registrierung von zwei Bildern ermöglicht, indem es ein Quellbild verformt, so dass es einem Zielbild ähnelt. 
Es stehen drei Betriebsarten zur Verfügung: 

\begin{enumerate}
\item ein vollautomatischer Modus; 
\item ein vollständig interaktiver Modus, bei dem die Verformung durch die Position einer beliebigen Anzahl von Landmarken eindeutig bestimmt ist; 
\item ein gemischter Modus, bei dem interaktive Landmarken nur verwendet werden, um eine ansonsten automatische Registrierungsprozedur anzuzeigen.
\end{enumerate}

Das Deformationsmodell besteht aus kubischen Splines, die Glätte und Vielseitigkeit gewährleisten. Das Registrierungskriterium enthält einen Vektor-Spline-Regularisierungstermin, um die Deformation physisch realistisch zu beschränken.\cite{unwrapj}

In dieser Anwendung wird es jedoch nicht zur Registrierung sondern nur zum erzeugen von Landmarken genutzt. Diese werden in eine Textdatei gespeichert, welche im programmierten Plug-In eingelesen und verwendet wird.


\subsubsection{ImageJ}
\newpage

\subsubsection{Eclipse}
einer offenen Entwicklungsplattform, die aus erweiterbaren Frameworks, Tools und Laufzeiten für den Aufbau, die Bereitstellung und das Verwalten von Software über den gesamten Lebenszyklus besteht.
\cite{eclipse}

\subsubsection{Java}