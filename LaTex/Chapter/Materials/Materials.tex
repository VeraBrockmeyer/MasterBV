%!TEX root = ..\..\main.tex

\section{Materialien}\label{sec:Materials}
\todo[inline, color=yellow]{Artjom}

\subsection{Hardware}\label{sec:Hardware}

Während des Entwicklungsphase wurde das Plugin auf den beiden Computern ausgeführt, die in den Tab. \ref{tab:Computer1} und Tab. \ref{tab:Computer2} beschreiben sind.

Beide Computer mussten die Software ausführen können die in Abs. \ref{sec:Software} beschreiben sind-

%Veras Computer:
\begin{table}[H]
	\centering
	\begin{tabular}{|l|l|}
		\hline
		\Absatzbox{}
		\textbf{Acer E5-571G}& \textbf{Description} \\
		\hline
		Processor & Intel Core i7 CPU @ 2.40 GHz\\
		\hline
		RAM & 8 GB  \\
		\hline 
		Graphic Card & NVIDIA GeForce 840M\\
		\hline
		Operating System &  Windows 10 Education 64 bit   \\
		\hline
	\end{tabular}
	\caption[Auszug aus dem Datenblatt des Acer E5-571G]{Auszug aus dem Datenblatt des Acer E5-571G}.
	\label{tab:Computer1}
\end{table}

%Artjoms Computer:
\begin{table}[H]
	\centering
	\begin{tabular}{|l|l|}
		\hline
		\Absatzbox{}
		\textbf{MSI GS70 2PE Stealth Pro}& \textbf{Beschreibung} \\
		\hline
		Prozessor & Intel Core i7 CPU @ $2.50\,$GHz \\
		\hline
		RAM & $8\,$GB \\
		\hline 
		Grafik-Karte 1 & Nvidia GeForce GTX 870M\\
		\hline
		Grafik-Karte 2 & Intel(R) HD Graphics 4600\\
		\hline
		Betriebssystem & Windows 8.1 64 bit \\
		\hline
	\end{tabular}
	\caption[Auszug aus dem Datenblatt des MSI GS70 2PE Stealth Pro]{Auszug aus dem Datenblatt des MSI GS70 2PE Stealth Pro}
	\label{tab:Computer2}
\end{table}

\subsection{Software}\label{sec:Software}

\subsubsection{Eclipse Entwicklungs-Umgebung}
Eclipse ist eine offene Software-Entwicklungsplattform, die aus erweiterbaren Frameworks, Tools und Laufzeit-Umgebungen für die Entwicklung, die Bereitstellung und das Verwalten von Software über den gesamten Lebenszyklus besteht.
\cite{eclipse}

Eclipse IDE for Java Developers Version: Neon.2 Release (4.6.2) wurde zur Entwicklung des ImageJ-Plugins in diesem Projekt genutzt, da es die Entwicklung von Java-Komponenten unterstützt. 


\subsubsection{Java}
Java ist eine objektorientierte Programmiersprache und gleichzeitig eine Laufzeitumgebung für diese. Ein besonderer Vorteil von Java gegenüber ähnlichen objektorientierten Sprachen wie z.B. C\texttt{++} ist die automatische Speicherverwaltung. \cite{java}

Das Plugin zur radialen Entzerrung von regelmäßigen Punktrastern wurde in der Programmiersprache Java (Version $ 1.8.0 31 $) geschrieben, damit es in ImageJ genutzt werden kann. ImageJ bietet bestimme Interface-Klassen an durch deren Implmentierung von ImageJ ausführbare Komponenten geschreiben werden können.

Java lief in der Laufzeitumgebung Java SE Runtime Environment (Version $ 1.8.0 31-b13 $), welche auf dem jeweiligen Betriebssystem (siehe Abs. \ref{sec:Hardware}) installiert war.

\subsubsection{\textit{ImageJ}}
\textit{ImageJ} ist eine plattformunabhängige Open-Source Bildbearbeitungssoftware, die in Java implementiert wurde. Mit \textit{ImageJ} können die gängigen Bildformate, wie TIF, JPEG oder GIF mit unterschiedlichen Bittiefen angezeigt, analysiert und bearbeitet werden~\cite{Collins_ImageJ}. Es ist weiterführend ein weitverbreitetes Werkzeug zur Entwicklung von Methoden und Algorithmen, die zur Analyse von Bilddaten vorgesehen sind. Die Vielzahl der Funktionen wird stetig von den Nutzern in Form von Plugins erweitert und verbessert, wie es in Open-Source-Plattformen gängige Praxis ist.

\subsubsection{ImageJ Makros}	
\textit{ImageJ} Makros~\cite{JMacros} sind kleine Programme, die eine Abfolge von \textit{ImageJ} Kommandos ausführen. Zur Erstellung dieser Programme wurde der Recorder entwickelt, welche die ausgeführten Kommandos als Textdatei abspeichert. Diese Textdateien können editiert werden und in \textit{ImageJ} beliebig oft ausgeführt werden. Sie stellen eine erhebliche Arbeitserleichterung beim Testen von Algorithmen mit mehreren Testdaten dar, da sie schnell implementiert sind. Vor allem werden sie genutzt um ein geeignetes Verfahren oder die richtige Parametrierung zu ermitteln. Ist ein richtiger Algorithmus gefunden, kann die Textdatei einfach in Java übersetzt werden.



\subsubsection{UnwarpJ}
\todo[inline, color=red]{Artjom}

UnwarpJ ist ein für ImageJ entwickeltes Plugin, das die elastische Registrierung von zwei Bildern ermöglicht, indem es ein Quellbild verformt, so dass es einem Zielbild ähnelt. 
Es stehen drei Betriebsarten zur Verfügung: 

\begin{enumerate}
\item ein vollautomatischer Modus; 
\item ein vollständig interaktiver Modus, bei dem die Verformung durch die Position einer beliebigen Anzahl von Landmarken eindeutig bestimmt ist; 
\item ein gemischter Modus, bei dem interaktive Landmarken nur verwendet werden, um eine ansonsten automatische Registrierungsprozedur anzuzeigen.
\end{enumerate}

Das Deformationsmodell besteht aus kubischen Splines, die Glätte und Vielseitigkeit gewährleisten. Das Registrierungs-Kriterium enthält einen Vektor-Spline-Regularisierungsterm, um die Deformation physisch realistisch zu beschränken.\cite{unwrapj}

In dieser Anwendung wird es jedoch nicht zur Registrierung sondern nur zum erzeugen von Landmarken genutzt. Diese werden in eine Textdatei gespeichert, welche im programmierten Plug-In eingelesen und verwendet wird.

\subsubsection{\textit{Apache Common Math} Bibliothek}
\todo[inline, color=red]{Artjom}
Commons Math in der Version $3.6.1$ ist eine Bibliothek von leichten, eigenständigen Mathematik- und Statistikkomponenten, die die häufigsten Probleme in der Java-Programmiersprache oder Commons Lang ansprechen.
\cite{appache}

In diesem Projekt wird der Levenberg-Marquadt Optimierer und die zugehörigen Klassen aus der Bibilothek verwendet um die Koeffizienten der radialen Verzerrung zu berechnen.

