%!TEX root = ../../main.tex"

\section{Materialien}\label{sec:Materials}
\todo[inline, color=red]{XXX}

\subsection{Hardware}\label{sec:Hardware}
\subsection{Software}

\subsubsection{Klassen}

\paragraph{point_grid_radial_affin_distor_}
Hauptklasse der Anwendung. Implementiert das Interface "PluginFilter" um über ImageJ aufgerufen werden zu können.

Die Klasse besitzt folgende Methoden und deren Funktion:

\begin{tabular}{p{0.6\textwidth} | p{0.5\textwidth}} 
\caption{Methoden der point_grid_radial_affin_distor_ Klasse}
run & Main-Methode des PlugIns in der die Optimierung aufgerufen wird\\
setup & Konstruktor-Methode des PlugIns in dem die Bildreferenz gespeichert wird\\
readData & Liest aus einer in ImageJ geöffneten Textdatei Punkt-Paare ein für Start- und Ziel-Koordinten\\
computeDrawRadialTransformation & \\
drawTargets & Zeichnet Punkte an den übergebenen Ziel-Koordinaten in das übergebene Bild\\
computeDrawAffineTransformation & \\
computeRadius2Center & Berechnet anhand der Parameter den Abstand zum Gittermittelpunkt\\
compute_radial_dist_koeff & Berechnet mit dem LevenbergMarquadt Optimierer die Koeffizienten der Radialen Verzerrung der übergebenen Punkt und gibt die Koeffizienten zurück\\
\end{tabular}

\paragraph{SimplePair}
Eine Einfache Klasse zum Speichern der Vorgabe- und Ziel Koordinaten und des Abstandes zum Mittelpunkt.

\paragraph{RadialDistFunction}
Klasse zum erzeugen der Funktionen für den Optimierer.

\begin{tabular}{p{0.6\textwidth} | p{0.5\textwidth}} 
\caption{Methoden der RadialDistFunction Klasse}
RadialDistFunction & Konstruktor der Klasse. Es wird ein SimplePoint Array erwartet welcher Koordinaten-Paare für Start- und Ziel-Koordinaten enthält.\\
realTargetPoints() & Gibt ein Array aus welches nur die Ziel-Koordinaten enthält. Dieses wird für den Optimierer benötigt.\\
retMVF & Funktion zur Modellierung der Radialen Verzerrung für den Optimierer. BErechnet zu den Vorgegeben Koeffizienten und einer Start-Koordinate die Ziel-Koordinate\\
retMMF & Jacobi-Matrix-Funktion zur Berechnung der Ableitung nach den einzelnen vom Optimierer vorgegebenen Koeffizienten \\
\end{tabular}

\subsubsection{ImageJ}
\newpage
