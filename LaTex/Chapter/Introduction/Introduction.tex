\section{Einleitung}\label{sec:Introduction}
\todo[inline, color=yellow]{Artjom}


Kameraobjektive erzeugen keine idealen sondern radial verzerrte Abbildungen. Um diese Verzerrung in den Abbildungen zu entfernen ist eine vorherige Berechnung der Objektiv-Verzerrung notwendig. 
Für eine solche Kamerakalibrierung wird ein regelmäßiges Gitter fotografiert. 
Mit Hilfe von Wertpaare aus den Koordinaten des regelmäßigen Gitters und der verzerrten Abbildung lasst sich die radiale Objektiv-Verzerrung mit Hilfe einer nichtlinearen Ausgleichsrechnung bestimmen. 
Diese Arbeit befasst sich mit der praktischen Umsetztung in Form eines ImageJ-Plugin, welches die unbekannten Koeffizienten der radialen Verzerrung berechnet und die zugehörige radial verzerrte Abbildung entzerrt.
























