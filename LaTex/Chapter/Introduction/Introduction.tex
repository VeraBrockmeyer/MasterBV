\section{Einleitung}\label{sec:Introduction}
\todo[inline, color=yellow]{Artjom}
Lineare Kameramodell wie das projektive Transformation bilden den Weltpunkt, Bildhauptpunkt und das Projektionszentrum auf dem selben optischen Strahl ab. Im idealen Fall werden nach der Transformation alle Bildgeraden wieder als Geraden dargestellt. In der Realität bilden gerade Kamerasysteme mit Weitwinkelobjektiven die Geraden gebogen ab, da die Brennweite klein ist und der Blickwinkel sehr groß~\cite{HartleyRadDist}. Ein weiterer Grund ist ein mangelhaftes Linsendesign mit einer fehlerhaften Krümmung der einzelnen Linsenelementen in einem Objektiv~\cite{WengRadDist}\cite{Zhang:1996:EGT:844381.845228}. Diese verzerrten Abbildungen werden in der Regel als kissenförmig beschrieben und Radiale Linsenverzerrung genannt. In einigen Fällen kann es aber auch zu tonnenförmigen Verzerrungen kommen. Neben den linearen Kameramodellen muss also für die vollständige Nachbildung einer realen Kamera auch die Linsenverzerrung berücksichtigt werden. 

Um diese Verzerrung in den Abbildungen zu entfernen ist eine vorherige Berechnung der Objektiv-Verzerrung notwendig. Für eine solche Kamerakalibrierung wird ein regelmäßiges Punktraster fotografiert aus einer Perspektive bei der Bildhauptpunkt und das Projektionszentrum senkrecht auf der optischen Achse liegen. 
Mit Hilfe von Wertpaare aus den Koordinaten des Punktrasters und der verzerrten Abbildung lässt sich die radiale Verzerrung mit Hilfe einer nichtlinearen Ausgleichsrechnung bestimmen. Die bekannte Levenberg-Marquart-Approximation(LMA) ist ein bewährtes Verfahren zur Lösung eines solchen nichtlinearen Ausgleichsproblem.

Diese Arbeit befasst sich in Kapitel~\ref{sec:System} mit der praktischen Umsetzung in Form eines ImageJ-Plugin, welches die unbekannten Koeffizienten der radialen Verzerrung annähert und die zugehörige radial verzerrte Abbildung entzerrt. In Kapitel~\ref{sec:Auswertung} werden die Ergebnisse dargestellt und diskutiert. Zusätzlich wird in Kapitel~\ref{Methode} auf die Grundlagen der LMA und der Radial-Distance-Function (RDF) im Detail erläutert.
























