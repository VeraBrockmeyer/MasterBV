\section{Einleitung}\label{sec:Introduction}
\todo[inline, color=red]{Vera}




\subsection{Motivation}\label{sec:Motivation}
\todo[inline, color=red]{Artjom}

Kameraobjektive erzeugen keine idealen sondern radial verzerrte Abbildungen. Um diese Verzerrung in den Abbildungen zu entfernen ist eine vorherige Berechnung der Objektiv-Verzerrung notwendig. Für eine solche Kamerakallibrierung wird ein regelmäßiges Gitter fotografiert. Mit Hilfe der Kenntnisse des regelmäßigen Gitters und der verzerrten Abbildung lasst sich die radiale Objektiv-Verzerrung berechnen. Diese Arbeit befasst sich mit einem ImageJ Plugin welches Koeffizienten der radialen Verzerrung berechnet und die zugehörige radial verzerrte Abbildung entzerrt.






















