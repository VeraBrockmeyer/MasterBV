\section{Abstrakt}
\todo[inline, color=yellow]{Vera}
Kameraobjektive erzeugen in der Regel radial verzerrte Abbildungen auf Grund von Konstruktionsfehlern. Um diese fehlerhafte Abbildungen zu korrigieren ist eine Berechnung der Objektiv-Verzerrung notwendig. Für eine solche Kamerakalibrierung wird zunächst ein regelmäßiges Gitter fotografiert. Im Anschluss werden mit Hilfe von Wertpaare aus den Koordinaten des regelmäßigen Gitters und der verzerrten Abbildung die radiale Verzerrung mit Hilfe einer nichtlinearen Ausgleichsrechnung bestimmt. Diese Arbeit befasst sich unter Anderem mit der praktischen Umsetzung in Form eines ImageJ-Plugin, deren Bewertung sowie den theoretischen Grundlagen.  





